\chapter{Konzeptionelle Betrachtung von Operational \& Real-Time BI}
\section{Definition und Einordnung}
\subsection{Business Intelligence}
Früher wie auch heute hat der Faktor Zeit höchste Bedeutung für den Konkurrenzkampf auf marktwirtschaftlich organisierten Märkten. Dabei spielt es keine Rolle ob es sich um die Entwicklung und Einführung neuer Produkte, deren Produktion oder den Vertrieb handelt. Unternehmen, die am schnellsten auf sich verändernde Bedingungen reagieren können, haben einen erheblichen und mächtigen Vorteil im Konkurrenzkampf mit anderen Unternehmen. \autocite[Vgl.][S. 1]{Stalk1988} Dieser Vorteil kann nur genutzt werden, wenn Unternehmen in die Lage versetzt werden, Entscheidungen schnell, präzise und richtig zu treffen. Um Entscheidungsträger in dieser Weise unterstützen, haben sich in den letzten Jahrzehnten \ac{BI}-Lösungen etabliert.

Erstmals taucht der Begriff \ac{BI} in dem Beitrag \enquote{A Business Intelligence System} von Hans Peter Luhn aus dem Jahr 1958 auf. Demnach handelt es sich bei \ac{BI} um ein automatisches und intelligentes System, dass Informationen für verschiedene Abteilungen automatisch zusammenfasst, um Aktionen daraus abzuleiten zu können. \autocite[Vgl.][S. 1]{Luhn1958ABI} Test
\subsection{Operational \& Real-Time BI}
\section{Vor- und Nachteile mit deren Auswirkungen in der Praxis}
