\chapter{Überführung der Konzepte - Case Study \enquote{Tesla-Produktion}}
Ziel dieser Arbeit ist es, die Konzepte Operational und Real-Time \ac{BI} anhand eines vereinfachten Beispiels zu veranschaulichen und Vor- und Nachteile abzuleiten. Hierzu wird eine hypothetische Case Study durchgeführt, in der eine Produktionsfirma simuliert wird. Aus den breiten Anwendungsmöglichkeiten der Konzepte wird sich für einen Produktionsbeispiel aufgrund der Annahme entschieden, dass dieses Beispiel für eine breite Menge an Personen leicht verständlich ist.
\section{Technische Umsetzung}
Kapitelbegründung --> Besondere Herausforderung Daten simulieren
\subsection{Konzeptionierung}
Ziel der Case Study ist es, die Vor- und Nachteile der gegebenen Konzepte zu verdeutlichen.
Um dies zu erreichen, wurde sich für die Simulation eines Produktionsprozesses - konkret Produktion von Tesla-Autos - entschieden.
\begin{figure}[H]
    \centering
    \includegraphics[width=\textwidth]{jreichwald-dhbw_wise_latex_template-fafad54e83ab/img/Prozessmodell.png}
    \caption{Strategisches Prozessmodell}
    \label{fig:bpmn}
\end{figure}
Am Strategischen Prozessmodell (Abb. \ref{fig:bpmn}) wird der gewählte (reduzierte) Prozess für die Simulation deutlich. Nach dem Bestelleingang, wird die Bestellung geprüft, geplant und läuft anschließend durch den eigentlichen Produktionsprozess. Jedes Auto muss in der Produktion vier Stufen durchlaufen, wobei jede Stufe von einer Maschine durchgeführt wird:
\begin{enumerate}
    \item Karosserie: Hier wird das Grundgerüst des Autos aufgebaut. Es kann zwischen Model S, Model 3 und Model X gewählt werden.
    \item Farbe: In dieser Station wird das Auto entsprechend lackiert. Es kann zwischen Pearl White Multi-Coat, Solid Black, Midnight Silver Metallic, Deep Blue Metallic und Red Multi-Coat gewählt werden
    \item Batterie: Es gibt vier Batterien, die abhängig vom Typ verbaut werden können: Standard Plus, Maximale Reichweite, Performance, Plaid
    \item Innenraum und Pilot: Zuletzt wird der Innenraum (schwarz oder weiß) und Teile für die Steuerung (Autonomes Fahren oder Assistenz) verbaut.
\end{enumerate} 
Das fertige Produkt wird anschließend eingelagert und, sobald der Lagerbestand an fertigen Erzeugnissen über 30 steigt, versandt. Man kann sich vorstellen, dass es logistisch sinnvoll ist, den Versandprozess erst ab einer gewissen Menge an Produkte anzustoßen.

Für die technische Umsetzung wurde als Datengrundlage eine SQL-basierte und damit relationale \ac{DB} geschaffen (MariaDB). Diese gilt im Rahmen der Simulation als Single-Point-of-Truth und repräsentiert damit den Zustand der Produktionshalle. Damit kann die \ac{DB} als logische Repräsentation oder Digital Twin des hypothetischen, \ac{CPS} \textit{Fabrik} betrachtet werden
\begin{figure}[H]
    \centering
    \includegraphics[width=0.7\textwidth]{jreichwald-dhbw_wise_latex_template-fafad54e83ab/img/ER-Diagramm.png}
    \caption{ER-Diagramm}
    \label{fig:er}
\end{figure}
In Anbetracht des ER-Diagramms (Abb. \ref{fig:er}) wird ebenfalls Bestellverarbeitungs- und Produktionsprozess (Abb. \ref{fig:bpmn}) deutlich: In Bestellung landen alle eingehenden Kundenbestellungen. Auf Basis dieser Bestellungen wird für jede Bestellung ein Auto generiert, dass der Kofiguration der Kundenbestellung entspricht, diesem zu produzierendem Auto werden Komponenten zugeordnet. Es werden also nur Bestellungen geplant, deren Konfiguration in Form von Komponenten im Lager zu finden ist und damit produziert werden kann. Komponenten, die bereits einem Auto zugeordnet sind, werden hierfür nicht berücksichtigt.
\\Das entsprechende Auto befindet sich damit in der Produktionspipeline und erhält den Status 0 (geplant).
In einer Produktionsstraße gibt es vier Maschinen, wobei jede dieser Maschinen eine individuelle Grundbearbeitungszeit erhält. Dies ist die Zeit, die eine Maschine für das Einbauen des Teils in genau ein Auto bei einer Produktivität von 100\% benötigt. Innerhalb dieser Zeit werden die dem Auto zugeordneten Komponenten verbaut. Damit wird der individuelle Wert des Autos erhöht und die Komponente aus dem Lager entfernt. Das Auto durchläuft dabei Status 1 - 4. Nach Durchlaufen der letzten Maschine wird das Auto zunächst eingelagert (Status 5). Der Bestand an fertigen Erzeugnissen wird erst dann ausgeliefert, wenn 30 Bestellungen abgearbeitet wurden. Nach Lieferung ist das Auto befindet sich das Auto im finalen Status 6.
Eine Übersicht der verschiedenen Status einer Auto-Instanz findet sich in Tabelle \ref{tab:status}.
\begin{table}[H]
    \centering
    \begin{tabular}{|c|l|l|}
     \toprule
     \hline
         \textbf{Status}& \textbf{Beschreibung} & \textbf{Bedeutung} \\ \hline
          0 & geplant & Auto Instanz auf Basiss der Bestellung erzeugt und\\
           &&Komponenten zugeordnet\\
           &&Wert ist 0\\ \hline
          1 & Karosserie & Befindet sich in M1 oder wartet auf M2\\
          &&Wert nach Einbau entspricht Typ Komponente\\ \hline
          2 & Farbe & Befindet sich in M2 oder wartet auf M3\\
          &&Wert nach Einbau entspricht Wert in 1 zzgl. Preis \\
          &&Farbe\\ \hline
          3 & Batterie & Befindet sich in M3 oder wartet auf M4\\
          &&Wert nach Einbau entspricht Wert in 2 zzgl. Preis \\
          &&Batterie\\ \hline
          4 & Innenraum & Befindet sich in M4\\
          &und Pilot&Wert entspricht Wert in 3\\ \hline
          5 & lagernd& Befindet sich im Lager\\
          &&Wert entspricht Summe der Komponentenpreise\\ \hline
          6 & geliefert& Außerhalb des Prozesses\\ \hline
    \bottomrule
    \end{tabular}
    \caption{Status und Wert Auto Instanz}
    \label{tab:status}
\end{table}
Neben den Information, die notwendig sind, um den Bestellverarbeitungs- und Produktionsprozess abbilden zu können, werden weitere Informationen in der \ac{DB} gespeichert. Diese dienen stellen Kennzahlen dar oder dienen der Berechnung dieser Kennzahlen im Sinne von Operational und Real-Time \ac{BI}. Näheres zu den Kennzahlen findet sich in den Abschnitten \ref{abs:anwOP} und \ref{abs:anwRT}.

\subsection{Produktions-Simulation und Kennzahlenberechung}
 Um Berichterstattung im Sinne von \ac{BI} zu ermöglichen wurde ein Simulationsskript geschaffen, das den Zustand der \ac{DB} in einem regelmäßigen Abstand erneuert\footnote{Der Gesamte Code ist unter \url{https://github.com/leventlukas/digital-boardroom.git} aufrufbar.}. Dadurch wird der Produktionsprozess widergespiegelt. Zusätzlich wurden mehrere Web-UIs geschaffen, die eine direkte Manipulation der \ac{DB} ermöglichen. Konkret kann hierüber eine Bestellung getätigt, der Lagerbestand an Rohstoffen erhöht und Einfluss auf den Maschinen-Status genommen werden. Im Folgenden wird auf für den weiteren Verlauf dieser Arbeit relevante Zeilen eingegangen:

\textbf{Nutzungsgrad}: Der Nutzungsgrad ist definiert als der Quotient der Produktiven Zeit durch die Gesamte Arbeitszeit. In jeder Iteration des Simulationsskriptes wird bestimmt, ob eine Maschine in der Zeit zur letzten Iteration gearbeitet hat. Hierfür wird geprüft, ob aktuell ein Auto in der Maschine ist. Wenn dies der Fall ist, wird Nutzung auf True gesetzt (Z. 212, 213), andernfalls
 auf False (Z. 337). Die Nutzung wird zum aktuellen Zeitpunkt mit der Dauer in Auslastung\_Maschine gespeichert. Zusätzlich wird der Leistungsgrad über die letzten zehn Einträge gespeichert. Konkret wird der Quotient der Summe aller Einträge für Dauer, mit Nutzung True, und der Summe aller Einträge für Dauer gesamt berechnet und eingetragen (Z. 321-333, 343, 433-438). Damit stellt Spalte Auslastung der Tabelle Auslastung\_Maschine den gleitenden Durchschnitt des Nutzungsgrades im zeitlichen Verlauf je Maschine dar.
 \begin{equation}\label{equ:nutz}
    Leistungsgrad_t = \frac{\sum_{i=t-10}^{t-1} Nutzungsdauer_i}{\sum_{i=t-10}^{t-1} Laufzeit_i}
\end{equation}

\textbf{Produktivität}: Die Produktivität beschreibt die prozentuale Leistungsfähigkeit einer Maschine in Relation zu ihrer Maximalen Leistungsfähigkeit und bestimmt damit die Bearbeitungszeit der Maschine. Sie kann direkt über das Web-UI \textit{Maschinenmanipulation} beeinflusst werden. Das Verhältnis von tatsächlicher Bearbeitungszeit und idealer Bearbeitungszeit in Abhängigkeit von der Produktivität kann durch Gleichung \ref{equ:prod} beschrieben werden. Die tatsächliche Bearbeitungszeit dient der Bestimmung, ob eine Maschine die Bearbeitung des Autos abgeschlossen hat (Z. 196, 618-629, 214-218). Über diesen Zusammenhang beeinflusst die Produktivität den Nutzungsgrad der Maschine.
\begin{equation}\label{equ:prod}
    t_{tatsächlich} = t_{ideal}*(p*0.01)^{-1}
\end{equation}

\textbf{Komponentenpreis}: Der Preis einer Komponente entspricht dem Einkaufspreis und definiert in Summe den Wert eines bestimmten Autos. Umgekehrt ist der Wert eines Autos durch die Summe der Preise der verbauten Komponenten definiert. Neue Komponenten können direkt über das Web-UI \textit{Bestellung} mit manipulierten Preisen in das Lager überführt werde. Hierüber wird indirekt Einfluss auf die Materialkosten der zu produzierenden Autos genommen (Z. 232-281).

\textbf{Marge}: Die Marge der aktuellen Produktionspipeline ist definiert als die Summe der Umsätze einer Bestellung in der Produktionspipeline abzüglich der Herstellungskosten der Autos (Deckungsbeitrag 1) durch den Umsatz. Als Produktionspipeline sind alle Autos definiert, deren Status kleiner 5 ist, also noch nicht eingelagert sind. Die Herstellungskosten setzen sich zusammen als die Summe der Komponentenpreise. Dieser Zusammenhang ist in den Gleichungen \ref{equ:HK} und \ref{equ:Ma} dargestellt. Hierüber beeinflusst der Preis der Komponenten direkt auf die Marge. Die Simulation geht von einer \ac{LiFo}-Systematik aus. Das bedeutet, dass die neusten Komponenten für die Planung aktueller Bestellungen verwendet werden (Z. 400-441). Die Berechnung erfolgt direkt über Abfrage von der \ac{DB} (Quelltext \ref{code:MSQL}).
\begin{equation}\label{equ:HK}
    Herstellungskosten_{Bestellung} = \sum Komponentenpreis_{Auto}
\end{equation}
\begin{equation}\label{equ:Ma}
    Marge_{Gesamt} = \frac{\sum Umsatz_{Bestellung} - \sum Herstellungskosten_{Bestellung}}{\sum Umsatz_{Bestellung}}
\end{equation}
\lstinputlisting[language=SQL, 
caption=Margenberechnung (SQL)
]{sql_marge.m} \label{code:MSQL}

\subsection{Architekturübersicht}
Die Übertragung der Konzepte Realtime-\ac{BI} und Operational-\ac{BI} in die vorgestellte Case Study lebt von drei Komponenten: 
\begin{enumerate}
    \item Das Simulationsskript dient dazu, in regelmäßigen Abständen den Zustand der \ac{DB} zu aktualisieren. Es werden ersterne Veränderungen in der Datenbank berücksichtigt und in einen konsistenten Zustand überführt. Hierbei wird auf den vorherigen Zustand zugegriffen und die Veränderung auf Basis der verstrichenen Zeit abgeleitet und zurück in die \ac{DB} geschrieben.
    \item Über Konfigurator und Simulator kann direkter Einfluss auf die \ac{DB} genommen werden: Es können Bestellungen getätigt, Maschinenproduktivität beeinflusst und der Lagerbestand zum aktuellen Marktpreis der Komponenten erhöht werden.
    \item Dieser Zustand ist dann Anhand des Dashboards einsehbar. Eine genaue Erläuterung der Daschboards und der Grafiken in Bezug auf ihre unternehmerische Bedeutung erfolgt in Abschnitt \ref{abs:anwOP}.
\end{enumerate}
\begin{figure}[h]
    \centering
    \includegraphics[width=0.99\textwidth]{jreichwald-dhbw_wise_latex_template-fafad54e83ab/img/Architektur-Vertikal.png}
    \caption{Architektur}
    \label{fig:arch}
\end{figure}
Technisch wurde sich bei der Simulationskomponente für eine Microservice-Architektur, die auf einem lokalen Kubernetes-Cluster läuft. Es gibt zwei Deployments, entschieden: Das \textit{digitalboardroom-server} deployment startet den Simulations-Server, der das Simulationsskript als Service bereitstellt. Diese Ressource ist repliziert und wird über einen NodePort-Service verfügbar gemacht. Der Service fungiert zusätzlich als LoadBalancer. Dieser Service wird vom \textit{digitalboardroom-loopclient} im gleichnamigen Deployment in einer Schleife alle 5 Sekunden konsumiert. Zusätzlich besteht die Möglichkeit den Service außerhalb des Clusters lokal zu adressieren. Alle Pods basieren auf dem gleichen Dockerimage, besitzen jedoch unterschiedliche Entrypoints. Image-Versionen werden über Dockerhub verwaltet.
\\Die Server-Replikas haben direkten Zugriff auf die \ac{DB} und sorgen für einen konsistenten Zustand.
Die \ac{DB}, sowie die Web-UI's werden über noez.de gehostet. Die gesamte Architektur findet sich in Abb. \ref{fig:arch}.
\textbf{UI-Tools und Warum}

\section{Anwendung Operational BI und Realtime BI} \label{abs:anwOP}
\section{Ergebnisse}
\textbf{verwendete Tools nicht praxisnahe --> andere Tool sind jedoch teuer deutlich teurer und komplexer}